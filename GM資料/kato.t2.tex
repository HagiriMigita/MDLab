\documentclass[a4j]{jarticle}
\renewcommand{\baselinestretch}{0.85}
\usepackage[top=1.5cm, bottom=1.5cm, left=1.5cm, right=1.5cm]{geometry}
\usepackage[dvipdfmx]{graphicx}
\usepackage{subcaption}
\usepackage{float}
\usepackage{booktabs}
\usepackage{listings,jvlisting}

\lstset{
  basicstyle={\ttfamily},
  identifierstyle={\small},
  commentstyle={\smallitshape},
  keywordstyle={\small\bfseries},
  ndkeywordstyle={\small},
  stringstyle={\small\ttfamily},
  frame={tb},
  breaklines=true,
  columns=[l]{fullflexible},
  numbers=left,
  xrightmargin=0zw,
  xleftmargin=3zw,
  numberstyle={\scriptsize},
  stepnumber=1,
  numbersep=1zw,
  lineskip=-0.5ex
}

\begin{document}

	\begin{flushright}
		MDLabグループミーティング資料\\
		25年12月02日(火)
	\end{flushright}

	\begin{center}
		{\Large	疑似ラベルを用いた遠赤外線画像からの物体検出}
	\end{center}

	\begin{flushright}
		{\large B4 加藤 達也}
	\end{flushright}

	\section{研究背景および目的}
	\begin{itemize}
		\item 背景: 完全自動運転の実用化に向けて技術の開発が進められており、その為に車載カメラ画像からの物体検出は重要な要素技術である。可視光画像からの物体検出は天候や時間帯によって精度が低下するので、その解決策として遠赤外線からの物体検出手法を考える。
		\item 課題:遠赤外線画像のデータセットは可視光画像のデータセットと比較して数が少ない。
		\item 目的: 遠赤外線画像を入力として低照度下でも安定的に動作する検出モデルを構築する。また、RGB画像に適応して得た検出領域を教師とするドメイン適応を用いて、遠赤外線領域における検出モデルを構築する。それらに加えて、データ拡張と損失関数の実装によって、より精度の高い検出モデルを構築する。
	\end{itemize}
	\begin{figure}[ht]
		%複数の図を並べて出力する方法%
		\centering
		\begin{minipage}[b]{0.3\columnwidth}
			\centering
			\includegraphics[height=3cm]{fig/FLIR_RGB_00002.jpg}%
			\caption{RGB画像}
		\end{minipage}
		\begin{minipage}[b]{0.3\columnwidth}
			\centering
			\includegraphics[height=3cm]{fig/FLIR_FIR_00002.jpg}%
			\caption{FIR画像}
		\end{minipage}
		\begin{minipage}[b]{0.35\columnwidth}
			\centering
			\includegraphics[height=3cm]{fig/domain_adaptation.png}%
			\caption{ドメイン適応の流れ}
		\end{minipage}
	\end{figure}
	\section{これまでの研究のまとめ}
	\subsection{データセットの更新}
	\begin{itemize}
		\item FLIR\_ADAS\_v1に加え新たなデータセットとしてFLIR\_ADAS\_v2とそれらを合わせたFLIR\_ADAS\_v1+v2を作成した
		\item v2では解像度・視野の補正、アノテーションの変換を行っているので、v1から画像の枚数は減ってしまったが、信頼性の高いデータが含まれている。結果として、v2はscoreのグラフが右に移動している。(図4、図5) 
		\item また、v1+v2ではデータ数が増加しているので、全体的にv1よりscoreのグラフが右に移動している。(図6)
		\item v1とv2では含まれるcarとpersonのアノテーションの数やそれぞれの大きさの個数が異なるので、学習や検出結果で異なる特性を持つ。
		\item v1+v2では画像のサイズが異なることから射影変換行列の切り替えが必要となる。実質現在はv1のみがデータとして有意にはたらいている。
	\end{itemize}
	\begin{figure}[H]
		\centering
		\begin{minipage}[H]{0.3\columnwidth}
			\centering
			\includegraphics[height=4cm]{fig/bbox_score_histgram_v1.png}
			\caption{v1のアノテーションの各スコア}			
		\end{minipage}
		\begin{minipage}[H]{0.3\columnwidth}
			\centering
			\includegraphics[height=4cm]{fig/bbox_score_histgram_v2.png}
			\caption{v2のアノテーションの各スコア}			
		\end{minipage}
		\begin{minipage}[H]{0.3\columnwidth}
			\centering
			\includegraphics[height=4cm]{fig/bbox_score_histgram_v1+v2.png}
			\caption{v1+v2のアノテーションの各スコア}			
		\end{minipage}
	\end{figure}
	\begin{table}[h]
		\centering
		\begin{minipage}{0.48\linewidth}
			\centering
			\begin{tabular}{|c|r|r|r|r|}
				\hline
				category & small & medium & large & total \\
				\hline
				person & 1762 & 2784 & 363 & 4909 \\
				car & 5611 & 6425 & 1417 & 13453 \\
				\hline
				all & 7373 & 9209 & 1780 & 18362 \\
				\hline
			\end{tabular}
			\caption{v1(sam2整形後のannotaionの数)}
		\end{minipage}
		\begin{minipage}{0.48\linewidth}
			\centering
			\begin{tabular}{|c|r|r|r|r|}
				\hline
				category & small & medium & large & total \\
				\hline
				person & 452 & 551 & 23 & 1026 \\
				car & 951 & 1121 & 350 & 2422 \\
				\hline
				all & 1403 & 1672 & 373 & 3448 \\
				\hline
			\end{tabular}
			\caption{v2(sam2整形後のannotationの数)}
		\end{minipage}
	\end{table}
	\begin{table}[ht]
		\centering
		\begin{tabular}{|c|r|r|r|r|}
			\hline
			version & カテゴリ & 平均 & 数 & 標準偏差 \\
			\hline
			v1 & person & 0.745 & 4,909 & 0.110 \\
				& car & 0.748 & 13,453 & 0.128 \\
			\hline
			v2 & person & 0.699 & 1,026 & 0.105 \\
				& car & 0.770 & 2,422 & 0.127 \\
			\hline
		\end{tabular}
		\caption{v1、v2の各カテゴリごとのアノテーションスコアの平均、数、標準偏差}
	\end{table}
	\subsection{損失関数の変更}
	\begin{itemize}
		\item 従来手法ではクラス、オブジェクトに対してBCELossが使用されていた、bboxにはIoULossが使用されていた。(YOLOXのデフォルトの損失関数)
		\item personのデータ数が少ない、また車と比べて検出精度が低いことからクラス不均衡に対して効果的に作用するFocalLossに変更、bboxはCIoULossに変更した。
	\end{itemize}
	\subsection{損失関数の最適化}
	\begin{itemize}
		\item Optunaを使用してYOLOXのデフォルトの損失関数と、提案手法における損失関数(FocalLossとCIoULoss)を各データセットに合わせての最適化を行った。
	\end{itemize}
	\section{前回のLTからの進捗}
		\subsection{データセットについて}
		\begin{itemize}
			\item 改めてデータセットの構造を理解した。
			\item v1では、RGB画像とThermal画像のサイズが異なり、射影変換を行う必要があったが、v2ではRGB画像をThermal画像のサイズにリサイズしている。その際に作成されるpair.jsonにはRGB画像とThermal画像のペア情報とリサイズ後に合わせたbbox情報も含まれている。なので、v2ではこれ以上緻密な射影変換を行う必要がない。
		\end{itemize}
		\subsection{損失関数の最適化について}
		\begin{itemize}
			\item Optunaを使用してYOLOXのデフォルトの損失関数と、提案手法における損失関数(FocalLossとCIoULoss)を各データセットに合わせての最適化を行った。v1、v2、v1+v2それぞれで最適なハイパーパラメータが異なることがわかった。これは各データセットでアノテーションの数や大きさの分布が異なることが影響していると考えられる。
			\item 各パラメータの比較一覧を作成した。
		\end{itemize}
		\subsection{各実験環境の結果の比較について}
			\subsection{共通の実験設定}
				\begin{itemize}
					\item ベースモデル:YOLOX-s
					\item フレームワーク:MMDetection\cite{MMDetection}
					\item 学習エポック数:300
					\item バッチサイズ:8
					\item クラス数:2(person, car)
					\item 最適化手法:SGD
					\item 評価指標:mAP(0.5:0.95)、mAP\_50、mAP\_70、mAP\_s、mAP\_m、mAP\_l
				\end{itemize}
			\subsubsection{チューニングによる検出結果への影響}
			\begin{table}[H]
				\centering
					\begin{minipage}{0.7\linewidth}
						\centering
						\begin{tabular}{|c|r|r|r|r|r|r|}
							\hline
							category & mAP & mAP\_50 & mAP\_70 & mAP\_s & mAP\_m & mAP\_l\\
							\hline
							person & 0.036 & 0.135 & 0.008 & 0.034 & 0.100 & 0.172 \\
							car & 0.274 & 0.552 & 0.245 & 0.121 & 0.534 & 0.721 \\
							\hline
						\end{tabular}
						\caption{v1(提案損失・チューニング前)のv1における検出結果}
					\end{minipage}
					\begin{minipage}{0.7\linewidth}
						\centering
						\begin{tabular}{|c|r|r|r|r|r|r|}
							\hline
							category & mAP & mAP\_50 & mAP\_70 & mAP\_s & mAP\_m & mAP\_l\\
							\hline
							person & 0.077 & 0.247 & 0.026 & 0.066 & 0.168 & 0.285 \\
							car & 0.359 & 0.672 & 0.347 & 0.226 & 0.513 & 0.774 \\
							\hline
						\end{tabular}
						\caption{v1(提案損失・チューニング後)のv1における検出結果}
					\end{minipage}
			\end{table}
			\begin{itemize}
				\item v1データセットにおいて、損失関数のチューニングを行うことで、全体的にmAPが向上した。 特にpersonクラスにおいて顕著な改善が見られた。 これはFocalLossがクラス不均衡に対して効果的に作用したためと考えられる。carに関してはmAP\_mが低下している。
			\end{itemize}
			\subsubsection{損失関数の違いによる検出結果への影響}
			\begin{table}[H]
				\centering
					\begin{minipage}{0.7\linewidth}
						\centering
						\begin{tabular}{|c|r|r|r|r|r|r|}
							\hline
							category & mAP & mAP\_50 & mAP\_70 & mAP\_s & mAP\_m & mAP\_l\\
							\hline
							person & 0.04 & 0.149 & 0.007 & 0.038 & 0.125 & 0.174 \\
							car & 0.32 & 0.61 & 0.292 & 0.134 & 0.56 & 0.778 \\
							\hline
						\end{tabular}
						\caption{v1(デフォルト損失・チューニング後)のv1における検出結果}
					\end{minipage}
					\begin{minipage}{0.7\linewidth}
						\centering
						\begin{tabular}{|c|r|r|r|r|r|r|}
							\hline
							category & mAP & mAP\_50 & mAP\_70 & mAP\_s & mAP\_m & mAP\_l\\
							\hline
							person & 0.077 & 0.247 & 0.026 & 0.066 & 0.168 & 0.285 \\
							car & 0.359 & 0.672 & 0.347 & 0.226 & 0.513 & 0.774 \\
							\hline
						\end{tabular}
						\caption{v1(提案損失・チューニング後)のv1における検出結果}
					\end{minipage}
			\end{table}
			\begin{itemize}
				\item v1データセットにおいて、提案損失関数を用いた場合、デフォルト損失関数と比較して全体的にmAPが向上した。 特にpersonクラスにおいて顕著な改善が見られた。 これはFocalLossがクラス不均衡に対して効果的に作用したためと考えられる。
				\item ただ、carに関してはmAP\_m、mAP\_lが低下している。これはv1データセットにおいてcarの中でもmediumが多かったので、FocalLossの効果が限定的であった可能性がある。
			\end{itemize}
			\subsubsection{データセットの違いによる検出結果への影響}
			\begin{table}[H]
				\centering
					\begin{minipage}{0.7\linewidth}
						\centering
						\begin{tabular}{|c|r|r|r|r|r|r|}
							\hline
							category & mAP & mAP\_50 & mAP\_70 & mAP\_s & mAP\_m & mAP\_l\\
							\hline
							person & 0.077 & 0.247 & 0.026 & 0.066 & 0.168 & 0.285 \\
							car & 0.359 & 0.672 & 0.347 & 0.226 & 0.513 & 0.774 \\
							\hline
						\end{tabular}
						\caption{v1(提案損失・チューニング後)のv1における検出結果}
					\end{minipage}
					\begin{minipage}{0.7\linewidth}
						\centering
						\begin{tabular}{|c|r|r|r|r|r|r|}
							\hline
							category & mAP & mAP\_50 & mAP\_70 & mAP\_s & mAP\_m & mAP\_l\\
							\hline
							person & 0.076 & 0.234 & 0.035 & 0.066 & 0.222 & 0.112 \\
							car & 0.404 & 0.753 & 0.374 & 0.291 & 0.542 & 0.76 \\
							\hline
						\end{tabular}
						\caption{v2(提案損失・チューニング後)のv1における検出結果}
					\end{minipage}
			\end{table}
			\begin{itemize}
				\item 上記の比較から分かることとしては、v2で学習したモデルではpersonのlargeの検出精度が低下してしまう。2.1の表で示した通り、v2データセットではlargeのpersonのアノテーション数が非常に少ないため、学習が十分に行われなかった可能性がある。person\_largeの割合としてv1は約7.4\%、v2は約2.2\%である。
				\item 一方でcarに関しては全体的にmAPが向上している。これについては同じく2.1のcarのアノテーションスコアの平均がv1から上がっていることが検出精度の向上に寄与していると考える。
			\end{itemize}
	\section{今後の課題\&スケジュール}
		\begin{itemize}
			\item v1+v2用損失関数のチューニングが終了次第、結果の比較を行う。
		\end{itemize}

\begin{thebibliography}{10}
	\bibitem{priorwork}谷本 樹希「セグメンテーションによる疑似ラベル補正を用いたドメイン適応型遠赤外線物体検出」 2024年修士論文
	\bibitem{FocalLoss}Tsung-Yi Lin, Priya Goyal, Ross Girshick, Kaiming He, Piotr Dollár Focal Loss for Dense Object Detection (7 Feb 2018)
	\bibitem{CIoULoss}Zhaohui Zheng, Ping Wang, Wei Liu, Jingdong Wang, Junjie Yan, and Dong Chen Distance-IoU Loss: Faster and Better Learning for Bounding Box Regression (6 Apr 2020)
	\bibitem{YOLOX}Ge Z, Liu S, Wang F, et al. YOLOX: Exceeding YOLO Series in 2021[C]//Proceedings of the IEEE/CVF Conference on Computer Vision and Pattern Recognition. 2021: 8370-8379.
	\bibitem{optuna}Takuya Akiba, Shotaro Sano, Takeru Yanase, Toshihiko Ohta, Masanori Koyama Optuna: A Next-generation Hyperparameter Optimization Framework (20 Feb 2019)
	\bibitem{FLIR}FLIR ADAS and Thermal Dataset. https://www.flir.com/oem/adas/adas-dataset-form/(accessed: 2025-11-22)
	\bibitem{SAM2} Nikhila Ravi, Valentin Gabeur, Yuan-Ting Hu, Ronghang Hu, Chaitanya Ryali, Tengyu Ma, Haitham Khedr, Roman Rädle, Chloe Rolland, Laura Gustafson, Eric Mintun, Junting Pan, Kalyan Vasudev Alwala, Nicolas Carion, Chao-Yuan Wu, Ross Girshick, Piotr Dollár, and Christoph Feichtenhofer. SAM 2: Segment Anything in Images and Videos. arXiv preprint arXiv:2408.00714, 2024.
	\bibitem{MMDetection} K. Chen, J. Wang, J. Pang, Y. Cao, Y. Xiong, X. Li, S. Sun, W. Feng,Z. Liu, J. Xu, Z. Zhang, D. Cheng, C. Zhu, T. Cheng, Q. Zhao, B. Li, X.Lu, R. Zhu, Y. Wu, J. Dai, J. Wang, J. Shi, W. Ouyang, C.C. Loy, and D.Lin, “MMDetection: Open MMLab detection toolbox and benchmark,” arXivpreprint, vol.arXiv:1906.07155, June 2019.
\end{thebibliography}
\end{document}
