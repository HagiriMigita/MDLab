\documentclass[a4j]{jarticle}
\renewcommand{\baselinestretch}{0.85}
\usepackage[top=1.5cm, bottom=1.5cm, left=1.5cm, right=1.5cm]{geometry}
\usepackage[dvipdfmx]{graphicx}
\usepackage{subcaption}
\usepackage{float}
\usepackage{booktabs}
\usepackage{listings,jvlisting}

\lstset{
  basicstyle={\ttfamily},
  identifierstyle={\small},
  commentstyle={\smallitshape},
  keywordstyle={\small\bfseries},
  ndkeywordstyle={\small},
  stringstyle={\small\ttfamily},
  frame={tb},
  breaklines=true,
  columns=[l]{fullflexible},
  numbers=left,
  xrightmargin=0zw,
  xleftmargin=3zw,
  numberstyle={\scriptsize},
  stepnumber=1,
  numbersep=1zw,
  lineskip=-0.5ex
}

\begin{document}

	\begin{flushright}
		MDLabグループミーティング資料\\
		25年12月02日(火)
	\end{flushright}

	\begin{center}
		{\Large	疑似ラベルを用いた遠赤外線画像からの物体検出}
	\end{center}

	\begin{flushright}
		{\large B4 加藤 達也}
	\end{flushright}

	\section{研究背景および目的}
	\begin{itemize}
		\item 背景: 完全自動運転の実用化に向けて技術の開発が進められており、その為に車載カメラ画像からの物体検出は重要な要素技術である。可視光画像からの物体検出は天候や時間帯によって精度が低下するので、その解決策として遠赤外線からの物体検出手法を考える。
		\item 課題:遠赤外線画像のデータセットは可視光画像のデータセットと比較して数が少ない。
		\item 目的: 遠赤外線画像を入力として低照度下でも安定的に動作する検出モデルを構築する。また、RGB画像に適応して得た検出領域を教師とするドメイン適応を用いて、遠赤外線領域における検出モデルを構築する。それらに加えて、データ拡張と損失関数の実装によって、より精度の高い検出モデルを構築する。
	\end{itemize}
	\begin{figure}[ht]
		%複数の図を並べて出力する方法%
		\centering
		\begin{minipage}[b]{0.3\columnwidth}
			\centering
			\includegraphics[height=3cm]{fig/FLIR_RGB_00002.jpg}%
			\caption{RGB画像}
		\end{minipage}
		\begin{minipage}[b]{0.3\columnwidth}
			\centering
			\includegraphics[height=3cm]{fig/FLIR_FIR_00002.jpg}%
			\caption{FIR画像}
		\end{minipage}
		\begin{minipage}[b]{0.35\columnwidth}
			\centering
			\includegraphics[height=3cm]{fig/domain_adaptation.png}%
			\caption{ドメイン適応の流れ}
		\end{minipage}
	\end{figure}
	\section{これまでの研究のまとめ}
	\subsection{データセットの更新}
	\begin{itemize}
		\item FLIR\_ADAS\_v1に加え新たなデータセットとしてFLIR\_ADAS\_v2とそれらを合わせたFLIR\_ADAS\_v1+v2を作成した
		\item v2では解像度・視野の補正、アノテーションの変換を行っているので、v1から画像の枚数は減ってしまったが、信頼性の高いデータが含まれている。結果として、v2はscoreのグラフが右に移動している。(図4、図5) 
		\item v1とv2では含まれるcarとpersonのアノテーションの数やそれぞれの大きさの個数が異なるので、学習や検出結果で異なる特性を持つ。
		\item v1+v2では画像のサイズが異なることから射影変換行列の切り替えが必要となる。実質現在はv1のみがデータとして有意にはたらいている。
	\end{itemize}
	\begin{table}[h]
		\centering
		\begin{minipage}{0.48\linewidth}
			\centering
			\begin{tabular}{c|rrrr}
				\hline
				category\_id & Small & Medium & Large & Total \\
				\hline
				person & 1762 & 2784 & 363 & 4909 \\
				car & 5611 & 6425 & 1417 & 13453 \\
				\hline
				all & 7373 & 9209 & 1780 & 18362 \\
				\hline
			\end{tabular}
			\caption{v1(sam2整形後のannotaionの数)}
		\end{minipage}
		\begin{minipage}{0.48\linewidth}
			\centering
			\begin{tabular}{c|rrrr}
				\hline
				category\_id & Small & Medium & Large & Total \\
				\hline
				person & 452 & 551 & 23 & 1026 \\
				car & 951 & 1121 & 350 & 2422 \\
				\hline
				all & 1403 & 1672 & 373 & 3448 \\
				\hline
			\end{tabular}
			\caption{v2(sam2整形後のannotationの数)}
		\end{minipage}
	\end{table}
	\subsection{損失関数の変更}
	\begin{itemize}
		\item 従来手法ではクラス、オブジェクトに対してBCELossが使用されていた、bboxにはIoULossが使用されていた。
		\item personのデータ数が少ない、また車と比べて検出精度が低いことからクラス不均衡に対して効果的に作用するFocalLossに変更。
		\item bboxはCIoULossに変更した。
	\end{itemize}
	\section{前回のLTからの進捗}
		\begin{itemize}
			\item v1からv2へのデータセットの更新という形ではなく、v1+v2でデータセットの拡張を図る方針に変更した。
			\item 谷本手法(v1)では、日中に撮影されたRGBとFIRで対応する画像が4358枚存在しており、それぞれ同じファイル名で一致させることで対応付けをしている。
			\item この形式でv2の画像も対応するRGBとFIRを同じファイル名にして対応付けを行った。
			\item 画像データの枚数が多いのでまだ実験終わってないです。
		\end{itemize}
	\section{今後の課題\&スケジュール}
		\begin{itemize}
			\item v1+v2の精度が確認でき次第、v1,v2の精度との比較を行う。
			\item 損失関数に関する理解を深め、より良い検出精度を目指す。できれば独自の計算で損失関数を作成したい。
		\end{itemize}

\begin{thebibliography}{10}
	\bibitem{FocalLoss}Tsung-Yi Lin, Priya Goyal, Ross Girshick, Kaiming He, Piotr Dollár Focal Loss for Dense Object Detection (7 Feb 2018)
	\bibitem{learning_rate}Zheng Ge,Songtao Liu,Feng Wang,Zeming Li,Jian Sun YOLOX: Exceeding YOLO Series in 2021 (6 Aug 2021)
\end{thebibliography}
	\begin{figure}[ht]
		\centering
		\begin{minipage}[b]{0.4\columnwidth}
			\centering
			\includegraphics[height=5cm]{fig/v1_score_histogram.png}
			\caption{v1のアノテーションの各スコア}			
		\end{minipage}
		\begin{minipage}[b]{0.4\columnwidth}
			\centering
			\includegraphics[height=5cm]{fig/v2_score_histogram.png}
			\caption{v2のアノテーションの各スコア}			
		\end{minipage}
	\end{figure}
\end{document}
